% \problem{}
% یک شرکت بیمه فرض می‌کند که هر فرد پارامتری تصادفی با توزیع پواسون و میانگین $\lambda$ دارد. پارامتر $\lambda$ به صورت متغیر تصادفی گاما با پارامترهای $\alpha$ و $k$ توزیع شده است. اگر فردی تازه بیمه شده باشد و در اولین سال $n$ تصادف مرتکب شود:
% \begin{enumerate}
%     \item تعداد متوسط تصادفات او در سال جاری را تعیین کنید.
%     \item مقدار شرطی پارامتر تصادف وی را پیدا کنید.
% \end{enumerate}

% يك فروشگاه دو مكان جداگانه براي كنترل مشتريان در زمان خروج دارد. اين مكانها هر كدام دو صندوق و دو كارمند براي كنترل مشتريان دارند. درصورتيكه X نشان دهنده تعداد دفعاتي باشدكهصندوقهادريكزمانمشخصدرمكان۱مورداستفادهقرارگرفتهاندو Yتعداد
% دفعات استفاده از صندوق ها در مكان ۲ باشد، تابع احتمال توام بصورت زير خواهد بود:
% y x012
% 0.04 0.04 0.12 0 0.05 0.19 0.08 1 0.30 0.12 0.06 2
% ۱( تابع چگالي حاشيه اي براي متعيير هاي X و Y و همچنين توزيع احتمال X در صورتي كه2= Yباشدرابيابيد
% ۲( )E(X|Y =2)،E(Xو)E(XYحسابكنيد. ۱
% fX|Y (x|y) = P{Y = y|X = x}f(x) P{Y = y}
% 
% پرسش سوم براي دو متغيير تصادفي X و Y تابع چگالي احتمال زير را در نظر بگيريد.
%  2 ( x + 2 y ) f(x,y) = 7
% 0
% 0 < x < 1 , 1 < y < 2 elsewhere.
% ۱( آيا X و Y مستقل هستند؟ ۱( )1 < P(X +Yحسابكنيد.
% ۳( )E(X +X2Yحسابكنيد. Y4
% ۴( اميد شرطي )E(X|Y = y حساب كنيد. پرسش چهارم
% ۱( ثابت كنيد دو متغير تصادفيX و Y )پيوسته( مستقل هستند اگر و تنها اگر تابع چگالي )جرم( احتمال مشترك آنها به صورت زير بيان شود:
% fX,Y (x, y) = h(x)g(y) − ∞ < x < ∞, −∞ < y < ∞
% ۲( اگر تابع چگالي توام X و Y به صورت زير باشد:
% f(x,y) = 6e−2xe−3y, 0 < x < ∞, 0 < y < ∞
% و خارج از اين ناحيه برابر صفر باشد، آيا متغيرهاي تصادفي مستقل هستند؟
% ۳( اگر تابع چگالي توام به صورت زير باشد، چه ميتوان گفت؟
% f (x, y) = 24xy, 0 < x < 1, 0 < y < 1, 0 < x + y < 1
% )خارج از اين ناحيه برابر صفر ميباشد(
% پرسش پنجم
% آمبولانسي با سرعت ثابت و در طول جادهاي به طول L حركت ميكند. در يك لحظه معين از زمان، حادثهاي در نقطهاي كه به تصادف روي جاده توزيع شده است رخ ميدهد. )يعني
% فاصلهاش از يك انتهاي ثابت جاده توزيع يكنواخت روي فاصله ]L ,0[ دارد.( همچنين فرض كنيد كه محل آمبولانس در لحظه حادثه نيز داراي توزيع يكنواخت است. توزيع فاصله آمبولانس
% از محل حادثه را با فرض استقلال حساب كنيد. ۲
% پرسش ششم 
% ۱( دو نقطه به تصادف روي خطي به طول 
% L انتخاب ميكنيم بهطوريكه دو نقطه در دو
% طرف از نقطه وسط خط قرار گيرند. به بيان ديگر، دو نقطه X و Y متغيرهاي تصادفي مستقل هستند بطوريكه X داراي توزيع يكنواخت روي فاصله ] L2 ,0[ و Y داراي توزيع
% يكنواخترويفاصله]L,L[است.احتمالاينكهطولبيندونقطهانتخابيبيشتراز L
% باشد را پيدا كنيد.
% ۲( دربخشقبل،احتمالاينكه۳پارهخطاز Xتا Y،از XتاL،واز YتاLبتوانندتشكيل اضلاع يك مثلث بدهند را پيدا كنيد. )توجه كنيد كه ۳ پارهخط ميتوانند تشكيل اضلاع
% يك مثلث را بدهند اگر طول هركدام از آنها كمتر از مجموع طول دو تاي ديگر باشد.(
% 22
% ۳

