\documentclass[12pt]{article}
\usepackage{../res/templates/HomeWorkTemplate}
\usepackage{circuitikz}
\usepackage[shortlabels]{enumitem}
\usepackage{hyperref}
\usepackage{tikz}
\usepackage{fontspec}
\usepackage{xepersian}
\usepackage{graphicx}
\usepackage{float}
\usepackage{changepage}

\usetikzlibrary{arrows,automata}
\usetikzlibrary{circuits.logic.US}
\settextfont[Scale = 1.0 ,
             BoldFont = *Bd ,
             ItalicFont = *It,
             BoldItalicFont = *BdIt,
             Extension =.ttf
            ]{XB Niloofar} 

%\settextfont{XB Niloofar.ttf}
\setdigitfont{XB Niloofar.ttf}
\newcounter{problemcounter}
\newcounter{subproblemcounter}
\setcounter{problemcounter}{1}
\setcounter{subproblemcounter}{1}
\newcommand{\problem}[1]
{
	\subsection*{
		پرسش
		\arabic{problemcounter} 
		\stepcounter{problemcounter}
		\setcounter{subproblemcounter}{1}
		#1
	}
}
\newcommand{\subproblem}{
	\textbf{\harfi{subproblemcounter})}\stepcounter{subproblemcounter}
}



\begin{document}

\handout
{احتمال و کاربرد آن}
{تمرین سری سوم}
{پریسا موسوی}
{ }

\problem{}
\subproblem{}
فرض کنید \( X_1, \dots X_n\) متغیرهای تصادفی مستقل با میانگین مشترک 
\( \mu \) و واریانس مشترک \( \sigma^2 \) باشند و 
\\ \( Y_i = X_i + X_{i+1} + X_{i+2} \). \\\\
برای \( j \geq 0 \)  و  \( 1 \leq i \leq n-2\) ،  \( \text{Cov}(Y_i, Y_{i+j}) \) را بدست آورید.
\problem{}
\subproblem{}
برای هر متغیر نامنفی $X$ و هر $a > 0$ ،
ثابت کنید\\
\[E[X] \geq aP[X\geq a]\]
\subproblem{}
با ایده اثبات بخش قبل، با فرض اینکه $X<2E[X]$
سعی کنید یک کران بالا برای مقدار $P[X \leq \dfrac{E[X]}{2}]$
بدست بیاورید.\\
\subproblem{}
ثابت کنید برای هر مقدار $t > 0$ که تابع مولد گشتاور در $X$
در $t$
قابل تعریف باشد و هر $a \in \mathbb{R}$
\[e^{-a.t}M_X(t) \geq P[X\geq a]\]\\

\textbf{پاسخ.}\\
\\
\textbf{آ)}
از تعریف امید ریاضی، می‌دانیم:
\[
E[X] = \int_0^a x f(x) \, dx + \int_a^\infty x f(x) \, dx.
\]

قسمت اول انتگرال مقداری نامنفی است، بنابراین داریم:
\[
E[X] \geq \int_a^\infty x f(x) \, dx.
\]

همچنین در قسمت دوم انتگرال داریم $x \geq a$ بنابرین:\\
\[
\int_a^\infty x f(x) \, dx \geq \int_a^\infty a f(x) \, dx.
\]

از اینجا نتیجه می‌گیریم:
\[
\int_a^\infty a f(x) \, dx = a \int_a^\infty f(x) \, dx = a P[X \geq a].
\]

در نتیجه:
\[
E[X] \geq a P[X \geq a].
\]

% اگر هر دو طرف نابرابری را بر \(a > 0\) تقسیم کنیم، داریم:
% \[
% P[X \geq a] \leq \frac{E[X]}{a}.
% \]

\textbf{ب)}\\

از تعریف امید ریاضی و با توجه به اینکه $0 \leq X \leq 2E[X] $، می‌دانیم:
\[
E[X] = \int_0^{\frac{E[X]}{2}} x f(x) \, dx + \int_{\frac{E[X]}{2}}^{2E[X]} x f(x) \, dx.
\]

عبارات داخل انتگرال مقادیری نامنفی است، بنابراین داریم:
\[
E[X] \leq {\frac{E[X]}{2}} \int_0^{\frac{E[X]}{2}} f(x) \, dx + 2E[X]\int_\frac{E[X]}{2}^{2E[X]} f(x) \, dx
\]
\[
= {\frac{E[X]}{2}} P[X \leq {\frac{E[X]}{2}}] + 2E[X] (1 - P[X \leq {\frac{E[X]}{2}}])
\]

با فرض $E[X] > 0$ و تقسیم دو طرف نامساوی به آن داریم:
\[
    1 \leq \frac{1}{2} P[X \leq {\frac{E[X]}{2}}] + 2 - 2P[X \leq {\frac{E[X]}{2}}]
\]

یا همان:\\
\[
P[X \leq {\frac{E[X]}{2}}] \leq \frac{2}{3} .
\]


\textbf{پ)}\\
روش اول)\\
برای اثبات نابرابری \[e^{-a \cdot t} M_X(t) \geq P[X \geq a],\] از تعریف تابع مولد گشتاور استفاده می‌کنیم:
\[
M_X(t) = E[e^{tX}] = \int_0^\infty e^{tx} f_X(x) \, dx.
\]

در بازه \(x \geq a\) داریم \(e^{tx} \geq e^{ta}\). بنابراین:
\[
M_X(t) \geq \int_a^\infty e^{ta} f_X(x) \, dx = e^{ta} P[X \geq a].
\]

دو طرف را بر \(e^{ta}\) تقسیم می‌کنیم:
\[
e^{-a \cdot t} M_X(t) \geq P[X \geq a].
\]

روش دوم )\\
برای حل این قسمت از \lr{markov's inequality} که در بخش اول اثبات کردیم استفاده می‌کنیم. داریم:
\[
	X \geq a \overset{t>0}{\Longleftrightarrow} e^{tX} \geq e^{ta}
\]
\[
	\overset{\textbf{\lr{markov's inequality}}}{\Longrightarrow} \mathbb{P}[e^{tX} \geq e^{ta}] \leq \frac{\mathbb{E}[e^{tX}]}{e^{ta}} = e^{-ta}\mathbb{E}[e^{tX}] = e^{-ta}M_X(t)
\]
توجه کنید $e^{tX}$ و $e^{ta}$ هردو نامنفی هستند بنابراین می‌توانیم از \lr{markov's inequality} استفاده کنیم. در اینجا حل مسئله کامل می‌شود. $\square$
\problem{}
$n$
توپ به طور همزمان در
$n$
جعبه انداخته میشود.
هر توپ به صورت تصادفی در یکی از جعبه هاقرار میگیرد. امید ریاضی تعداد جعبه هایی که خالی خواهند ماند را به دست اورید.

\problem{}
فرض کنید $X$ یک متغیر تصادفی
نرمال استاندارد 
و $Y$ یک متغیر مستقل از $X$ با تابع جرم احتمال
زیر باشد 
\[
    P(Y = +1) = P(Y = -1) = \frac{1}{2}
\]\\
\subproblem{}
ثابت کنید $Z = XY$ هم توزیع نرمال دارد.\\
\subproblem{}
میانگین و واریانس $Z$ را بیابید.\\
\subproblem{}
$cov(X,Y)$ را بدست آورید.\\
\subproblem{}
آیا $X+Z$ هم توزیع نرمال دارد؟\\


\textbf{پاسخ.}\\
\\
\textbf{آ)}\\
با تابع گشتاور مسئله را حل می‌کنیم:
\[
	M_Z(t)=\mathbb{E}[e^{tZ}]=\mathbb{E}[e^{tXY}]=\sum_{y\in \{+1,-1\}} \mathbb{E}[e^{tXy}]\mathbb{P}[Y=y]
\]
\[
	\Longrightarrow M_Z(t)=\frac{1}{2}\mathbb{E}[e^{tX}]+\frac{1}{2}\mathbb{E}[e^{-tX}] = \frac{1}{2}M_X(t)+\frac{1}{2}M_X(-t) \overset{X\sim \mathcal{N}(0,1)}{=} \textbf{\lr{exp}}(\frac{t^2}{2})
\]
پس توزیع $Z$ نیز نرمال استاندارد است.

\textbf{ب)}\\
طبق قسمت قبل با توجه به نرمال استاندارد بودن توزیع $Z$ به ترتیب $\mu_Z = 0$ و $\sigma_{Z}^2 = 1$ محاسبه می‌شود.

\textbf{پ)}\\
\[
		Cov(X,Y) = \mathbb{E}[XY]-\mathbb{E}[X]\mathbb{E}[Y] = \mathbb{E}[Z] - \mathbb{E}[X]\mathbb{E}[Y] = 0 - 0 \times 0 = 0
\]

\textbf{ت)}\\

\[
	M_{X+Z}(t) = \mathbb{E}[e^{t(X+Z)}] = \mathbb{E}[e^{tX(1+Y)}] = \sum_{y\in \{+1,-1\}} \mathbb{E}[e^{tX(1+y)}]\mathbb{P}[Y=y]
\]
\[	
	= \frac{1}{2}\mathbb{E}[e^{tX(1+1)}] + \frac{1}{2}\mathbb{E}[e^{tX(1-1)}] = \frac{1}{2}\mathbb{E}[e^{2tX}] + \frac{1}{2}\mathbb{E}[e^{0}] = \frac{1}{2}M_X(2t) + \frac{1}{2} = \frac{1}{2}\textbf{\lr{exp}}(2t^2) + \frac{1}{2}
\]
می‌دانیم اگر $W \sim \mathcal{N}(\mu,\sigma^2)$ باشد داریم:
\[
	M_W(t) = \textbf{\lr{exp}}(\mu t + \frac{\sigma^2t^2}{2})
\] 
حال طبق قضیه یکتایی ثابت می‌شود که $X+Z$ نرمال نیست و حل مسئله به اتمام می‌رسد. $\square$

% \subsection*{(الف) اثبات کنید \( Z = XY \) توزیع نرمال دارد.}

% از آنجایی که \( Y \) مقدارهای \( +1 \) و \( -1 \) را با احتمال برابر می‌گیرد و مستقل از \( X \) است:
% \[
% Z = XY \quad \text{مقدارهای} \pm X \ \text{را می‌گیرد.}
% \]
% توزیع \( Z \) توسط توزیع شرطی \( Z \) به شرط \( Y \) مشخص می‌شود:
% \[
% Z \mid (Y = +1) = X \quad \text{و} \quad Z \mid (Y = -1) = -X.
% \]
% با توجه به اینکه \( P(Y = +1) = P(Y = -1) = \frac{1}{2} \) و \( X \sim N(0, 1) \)، \( Z \) حول صفر متقارن است. همچنین:
% \[
% f_Z(z) = \frac{1}{2} f_X(z) + \frac{1}{2} f_X(-z),
% \]
% که در آن \( f_X(x) = \frac{1}{\sqrt{2\pi}} e^{-x^2 / 2} \) است. چون \( f_X(-z) = f_X(z) \)، نتیجه می‌گیریم:
% \[
% f_Z(z) = f_X(z).
% \]
% پس \( Z \sim N(0, 1) \).

% \subsection*{(ب) میانگین و واریانس \( Z \)}

% 1. میانگین \( Z \):
% \[
% \mathbb{E}[Z] = \mathbb{E}[XY] = \mathbb{E}[X] \mathbb{E}[Y] \quad \text{(با توجه به استقلال \( X \) و \( Y \))}.
% \]
% \[
% \mathbb{E}[X] = 0 \quad \text{و} \quad \mathbb{E}[Y] = 0،
% \]
% بنابراین:
% \[
% \mathbb{E}[Z] = 0.
% \]

% 2. واریانس \( Z \):
% \[
% \text{Var}(Z) = \mathbb{E}[Z^2] - (\mathbb{E}[Z])^2.
% \]
% \[
% Z^2 = (XY)^2 = X^2Y^2 \quad \text{و} \quad Y^2 = 1،
% \]
% پس:
% \[
% \text{Var}(Z) = \mathbb{E}[X^2] = \text{Var}(X) = 1.
% \]

% \subsection*{(پ) کوواریانس \( X \) و \( Y \)}

% کوواریانس به صورت زیر است:
% \[
% \text{Cov}(X, Y) = \mathbb{E}[XY] - \mathbb{E}[X]\mathbb{E}[Y].
% \]
% چون \( \mathbb{E}[X] = 0 \) و \( \mathbb{E}[Y] = 0 \)، داریم:
% \[
% \text{Cov}(X, Y) = \mathbb{E}[XY].
% \]
% با توجه به استقلال \( X \) و \( Y \):
% \[
% \text{Cov}(X, Y) = \mathbb{E}[X]\mathbb{E}[Y] = 0.
% \]

% \subsection*{(ت) آیا \( X + Z \) توزیع نرمال دارد؟}

% می‌دانیم:
% \[
% X + Z = X + XY = X(1 + Y).
% \]
% از آنجایی که \( Y \) مقدارهای \( +1 \) و \( -1 \) را می‌گیرد، \( 1 + Y \) مقدارهای \( 0 \) و \( 2 \) را با احتمال برابر می‌گیرد. بنابراین:
% \[
% X + Z =
% \begin{cases}
% 0 & \text{اگر } Y = -1, \\
% 2X & \text{اگر } Y = +1.
% \end{cases}
% \]
% این یک ترکیب توزیع‌ها است و \textbf{توزیع نرمال ندارد}.

% \end{document}

\problem{}
فرض کنید \( X \) یک متغیر تصادفی نمایی با پارامتر \( \lambda \) باشد و \( Y \) متغیر تصادفی جزء اعشاری \( X \) باشد. (جزء اعشاری \( X \) به صورت \( X - \lfloor X \rfloor \) تعریف می‌شود.)
\\
\\
\subproblem{}
توزیع \( Y \) را محاسبه کنید.
\\
\subproblem{}
امید ریاضی \( Y \) را محاسبه کنید.
\\
\subproblem{}
واریانس \( Y \) را محاسبه کنید.

\problem{}
فرض کنید $\{X_n\}_n$ دنباله ای از متغیر های تصادفی پوآسون با پارامتر $1$ و مستقل باشد، با استفاده از این دنباله و قضیه حد مرکزی نشان دهید
:\\
\[
    \lim_{n \to \infty}\dfrac{1}{e^n}\sum_{k=0}^{n} \dfrac{n^k}{k!}=\dfrac{1}{2}
\]
\problem{}
\begin{enumerate}
    \item روی خطی به طول $L$، دو نقطه تصادفی $X$ و $Y$ به طور مستقل انتخاب شده‌اند. $X$ روی بازه $[0, \frac{L}{2}]$ و $Y$ روی بازه $[\frac{L}{2}, L]$ یکنواخت توزیع شده است. احتمال اینکه فاصله بین این دو نقطه بیش از $\frac{L}{2}$ باشد، پیدا کنید.
    \item احتمال اینکه $X$، $Y$، و نقطه $L$ اضلاع یک مثلث تشکیل دهند را تعیین کنید (طول هر ضلع باید از مجموع طول دو ضلع دیگر کمتر باشد).
\end{enumerate}


\end{document}