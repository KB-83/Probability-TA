\problem{}
یک جمع \( n \) نفری تصمیم به بازی اسم فامیل می‌گیرند. هر فرد به صورت مستقل از دیگران و بدون تقلب با آن‌ها کلمات را می‌نویسد. هر دور این بازی زمانی پایان می‌یابد که اولین نفر تمام کلمات را بنویسد.
\\
\\
\subproblem{}
اگر زمانی که طول می‌کشد تا نفر \( i \)ام کلمات را بنویسد، از توزیع \( \text{Exp}(\lambda_i) \) پیروی کند، امید ریاضی و واریانس طول هر دور از این بازی چقدر است؟
\\
\subproblem{}
پریسا و کژال می‌خواهند امتیازات بازی را محاسبه کنند. 
اما از آن‌جایی که پریسا و کژال هیچ‌کدام حرف دیگری را قبول ندارند، تصمیم می‌گیرند هر دو امتیازات را
 محاسبه کنند و در نهایت اعدادی که به دست می‌آورند را با هم مقایسه کنند.
اگر زمانی که پریسا و کژال نیاز دارند
 تا امتیازات همه‌ی افراد را جمع بزنند به ترتیب \( P \) و \( K \) باشند و 
 از توزیع‌های \( P \sim \text{Exp}(a) \) و \( K \sim \text{Exp}(b) \) 
 تبعیت کنند، به طور میانگین بعد از پایان یک دور
 ، شمردن امتیازات چقدر طول می‌کشد؟ (در صورت نیاز، فرض کنید \( a > b \)
  است؛ بالاخره پریسا خیلی سریع‌تر از کژال حساب می‌کند!)
