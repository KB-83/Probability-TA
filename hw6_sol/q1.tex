\problem{}
فرض کنید $n$ آزمایش برنولی به صورت مستقل با احتمال موفقیت $\frac{1}{2}$
اجرا میشود.
نشان دهید اگر در نظر بگیریم $X$ متغیر تصادفی تعداد موفقیت های 
حاصل از این آزمایش ها باشد، داریم:\\
\[ Pr[X \geq \frac{n}{2} + \sqrt{n}] \leq \frac{1}{4}\]
\textbf{پاسخ.}
فرض کنید \( X_1, X_2, \ldots, X_n \) متغیرهای تصادفی برنولی با احتمال موفقیت \( p = \frac{1}{2} \) باشند و مجموع آن‌ها را به صورت \( X = \sum_{i=1}^n X_i \) تعریف کنیم. همچنین داریم:\\
\[
\mathbb{E}[X_i] = p = \frac{1}{2}, \quad \text{Var}(X_i) = p(1-p) = \frac{1}{4}.
\]

% ### واریانس مجموع

واریانس مجموع متغیرهای تصادفی برابر است با:
\[
\text{Var}(X) = \sum_{i=1}^n \text{Var}(X_i) + 2 \sum_{1 \leq i < j \leq n} \text{Cov}(X_i, X_j).
\]

که در این مثال $\text{Cov}(X_i, X_j) = 0 \quad \text{  for  } i \neq j$ پس داریم:\\
\[
\text{Var}(X) = \sum_{i=1}^n \text{Var}(X_i) = n \cdot \frac{1}{4} = \frac{n}{4}.
\]

% ### اعمال نابرابری چبیشف

برای محاسبه \( Pr\left(X \geq \frac{n}{2} + \sqrt{n}\right) \)، از نابرابری چبیشف استفاده می‌کنیم. نابرابری چبیشف به صورت زیر تعریف می‌شود:
\[
Pr(|X - \mathbb{E}[X]| \geq k\sigma(X)) \leq \frac{1}{k^2}.
\]
همچنین نامساوی بالا به ما نامساوی زیر را نیز نتیجه میدهد:\\
\[
Pr(X - \mathbb{E}[X] \geq k\sigma(X)) \leq \frac{1}{k^2}.
\]
با قرار دادن $k=2$ و جایگذاری مقادیر بدست آمده تا اینجا داریم:\\
\[
Pr(X - \frac{n}{2} \geq \sqrt{n}) \leq \frac{1}{4}.
\]
که همان خواسته سوال است.$\square$